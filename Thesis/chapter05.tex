\chapter{Conclusion and Future Work} \label{chapter5}
This research has shown that it is possible to predict failure in modern
\ac{MS} enterprise authentication architecture given a representative fault
load.  While two out of the three fault loads introduced in this research were
not successful in generating useful failure, one was.  More importantly,
without modification, the original \ac{AFP} was incapable of inducing failure
that was useful in training a statistical learning model.  

\section{Future Work}
Future lines of research should include the following\dots

\begin{itemize}
\item{More automation using VMWare APIs}
\item{Implement more of the operators from G-SWFIT and better automate
injection/training phases}
\item{Automate the event checking process... need a way to better determine
when underlying system has changed}
\item{Implement more predictors?}
\item{Better define method for determining when to run AFP? Sliding time
window?  Machine learngin?}
\item{Continuously running AFP?  Let it run continuously in the background to
capture new failures.  The same way our tool reports failure, have a health
checking daemon running in the background that will report to syslog when
failure has occured so that it gets labelled}
\item{More data.  Use variables as recommended by Irrera et al.}
\item{Realistic data.  Get real failure data from 83/561 NOS}
\item{Implement and use the AFP to predict failure in production environment!}
\item{More features.  Solve volume/velocity problem using solutions like STORM,
CAPSA, SPARK, and OS Query}
\item{Try new fault-loads on another service (like web-server)}
\item{This works in a small environment, but may face challenges when scaling}
\item{Explore fault injection in processes outside targetted area.  Won't find
faults in target, but may reproduce faults in overall system}
\end{itemize}

\section{Conclusion}
This research explored the use of the AFP to predict failure in MS domain
controllers... unaltered, did not work... not enough lead time.

Memory leak did provide useful results if the user isn't concerned with false
positives.  Noticed that there are axiomatic predictors in log messages as well
that may precede SVM appraoch.

CPU leak did not provide any useful information.  Failure was not achieved.
Service just became very slow.

Heap space was not very consistent.  Domain controller will go back to disk if
heap is corrupted.  Maybe in the future, consider corrupting both disk and
memory.

Bottom line: when a process is designed to handle these types of faults, then
failure prediction will not be possible.

**Still recommend using all fault loads in automated process... while the
software may not be vulnerable to them now, a software update could get pushed
which introduces this vulnerability.
