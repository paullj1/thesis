\chapter{Conclusion and Future Work} \label{chapter5}
This chapter outlines several lines of future work based on the outcomes of
this research.  The future work is then followed by the conclusions drawn from
this work and a discussion of their impact.

\section{Future Work}
Several lines of research following this work are presented in this section.
First and foremost, in order to put this technique into use on production
systems, the proof of concept \ac{W-SWFIT} application must be completed.
Furthermore, while automation was a consideration while conducting this
research, it was not implemented.  To be effective in a production environment,
the entire \ac{AFP} process must be automated.

One especially relevant and interesting line of effort that should follow this
work is to better identify when the underlying system has changed enough to
require retraining.  While the process is automated, it will unlikely be
necessary after every software update.  In order to avoid unnecessary use of
resources, this process could be explored.

As was demonstrated with the boosted decision trees, other statistical
classifiers could be explored.  The \ac{AFP} is not limited to a single
predictor~\cite{irrera2015}.  A series of prediction models can be used to vote
on the state of a system, the output being the majority.  In addition to
exploring other statistical learning models, additional states (or classes)
could be explored.  For example, instead of a failure state, a classification
model could be used to predict when a system would be idle to know when best to
install software updates.  Further, a classification model may be able to
automate the classification and prediction of when a target was under a
malicious attack in a method similar to the \ac{AFP}.

Since the original case study was conducted against a web server, one line of
research that could follow this work would be to explore the new fault-loads on
a service like the \ac{MS} web server: \ac{IIS}.  

Also, the use of additional features such as system health and diagnostic
information to predict failure could be used.  While it has been reported that
these features do not perform well on their own~\cite{salfnerSurvey}, when used
in conjunction with reported errors or log messages results could improve.
Unfortunately, the use of more data like this in enterprise systems may not be
possible given the increase in volume and velocity.  However, tools for dealing
with these problems have been improving in recent years~\cite{meng2016}.

Finally, the integration of actual failure data with the \ac{AFP} should be
explored.  Bootstrapping could be used to better integrate an actual failure
that occurs into the training phase, but actual failure data should always be
preferred over simulated data.

\section{Conclusion}
This research explored the use of the \ac{AFP} with additional fault loads to
predict failure in the \ac{MS} \ac{DC}s.  It has been shown that it is possible
to predict failure in modern \ac{MS} enterprise authentication architecture
given a representative fault load.  Unfortunately, at the time of writing, two
out of the three fault loads introduced in this research were not successful in
generating useful failure.  Additionally, unmodified fault injection as was
used in the original \ac{AFP} implementation, does not induce failure in
\ac{MS} \ac{DC}s that can be used to train a statistical learning model.  These
new fault loads are not useless however.  As was demonstrated with the \ac{SVM}
predictor, the underlying system changes can introduce or eliminate an
applications vulnerability to certain types of faults.  For this reason, if the
\ac{AFP} is implemented on \ac{MS} \ac{DC}s, all fault loads should be used in
the execution and training phases.

In addition to the new fault loads introduced in this work, a load generator
has also been presented:  \ac{D-PLG}, capable of sufficiently simulating peak
usage of a \ac{MS} enterprise \ac{DC}.  Additional uses for \ac{D-PLG} outside
of use in the \ac{AFP} framework include capacity planning/sizing, network
security testing and auditing, and software testing.  This research also
introduced \ac{W-SWFIT} which can be used to perform fault injection for a
variety of additional uses like software testing and auditing.

In conclusion, the modified \ac{AFP} framework as presented here can be used to
effectively predict failures that might occur in a production environment and
is capable of adapting to underlying system changes.  For this reason, it is
recommended that if the \ac{AFP} framework is to be implemented as laid out in
this research, all fault loads should be integrated to maximize the frameworks
ability to adapt to these changes.  Finally, real failure data is difficult to
obtain given how rare failure is in modern systems.  Unfortunately, even after
it is obtained, it can rapidly become deprecated by underlying system changes.
Using the \ac{AFP} to generate simulated failure data is the next best thing
which provides more useful predictions than are available with no failure data.
