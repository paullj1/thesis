\chapter{Conclusion and Future Work} \label{chapter5}
This chapter outlines several lines of future work based on the outcomes of
this research.  The future work is then followed by the conclusions drawn from
this work and a discussion of their impact.

\section{Future Work}
Several lines of research following this work are presented in this section.
First and foremost, in order to put this technique into use on production
systems, the proof of concept \ac{W-SWFIT} application must be completed.
Furthermore, while automation was a consideration while conducting this
research, it was not implemented.  To be effective in a production environment,
the entire \ac{AFP} process must be automated.

One especially relevant and interesting line of effort that should follow this
work is to better identify when the underlying system has changed enough to
require retraining.  While the process is automated, it will unlikely be
necessary after every software update.  In order to avoid unnecessary use of
resources, this process could be explored.

As was demonstrated with the boosted decision trees, other statistical
classifiers could be explored.  The \ac{AFP} framework is not limited to a
single predictor~\cite{irrera2015}.  A series of prediction models can be used
to vote on the state of a system, the output being the majority.  In addition
to exploring other statistical learning models, additional states (or classes)
could be explored.  For example, instead of a failure state, a classification
model could be used to predict when a system would be idle to know when best to
install software updates.  Further, a classification model may be able to
automate the classification and prediction of when a target was under a
malicious attack in a method similar to the \ac{AFP} framework.

An additional area of exploration should be to combine the newly developed
fault loads and reported error methodology to the monitoring approach used by
Irrera, et al.~\cite{irrera2015} using the \emph{Logman} tool to collect
additional system information to train a predictor.  While it has been reported
that these features do not perform well on their own~\cite{salfnerSurvey}, when
used in conjunction with reported errors or log messages, results could
improve.  Unfortunately, while tools for dealing with increased volume and
velocity have improved recently~\cite{meng2016}, the use of more data like this
in enterprise systems may not be possible given storage, geographic, and
processing constraints.

Finally, the integration of actual failure data with the \ac{AFP} framework
should be explored.  Bootstrapping could be used to better integrate actual
failure data into the training phase. 

\section{Conclusion}
This research explored the use of the \ac{AFP} framework with additional fault
loads to predict failure using reported errors in the \ac{MS} \ac{DC}s.  It has
been shown that it is possible to predict failure in modern \ac{MS} enterprise
authentication architecture given a representative fault load.  Unfortunately,
at the time of writing, two out of the three fault loads introduced in this
research were not successful in generating useful failure data.  Additionally,
unmodified fault injection as was used in the original \ac{AFP} framework
implementation, does not induce failure in \ac{MS} \ac{DC}s that can be used to
train a statistical learning model using reported errors.  These new fault
loads are not useless however.  As was demonstrated with the \ac{SVM}
predictor, the underlying system changes can introduce or eliminate an
applications vulnerability to certain types of faults.  For this reason, if the
\ac{AFP} framework is implemented on \ac{MS} \ac{DC}s, all fault loads should
be used in the execution and training phases.

In addition to the new fault loads introduced in this work, a load generator
has also been presented:  \ac{D-PLG}, capable of sufficiently simulating peak
usage of a \ac{MS} enterprise \ac{DC}.  Additional uses for \ac{D-PLG} outside
of use in the \ac{AFP} framework include capacity planning/sizing, network
security testing and auditing, and software testing.  This research also
introduced \ac{W-SWFIT} which can be used to perform fault injection for a
variety of additional uses like software testing and auditing.

The impact of this research should not be over estimated.  As mentioned, a
major limitation of this technique is that it is not able to predict malicious
acts or \emph{Act of God} events.  Furthermore, the data generated are still
simulated data and as such, may not completely capture all possible failure
events.  The \ac{AFP} framework as presented here will however provide more
reliable predictions than are currently available today.

In conclusion, the modified \ac{AFP} framework as presented here can be used to
effectively predict failures that might occur in a production environment and
is capable of adapting to underlying system changes using only reported errors.
For these reasons, it is recommended that if the \ac{AFP} framework is to be
implemented as laid out in this research, all fault loads should be integrated
to maximize the frameworks ability to adapt to these changes.  To improve the
efficacy of a predictor trained using this generated data, real failure data
and additional predictors can easily be integrated if available.  Finally, real
failure data is difficult to obtain given how rare failure is in modern
systems.  Unfortunately, even after it is obtained, it can rapidly become
deprecated by underlying system changes.  Using the \ac{AFP} to generate
simulated failure data is the next best thing to having real data and provides
more useful predictions than are available with no failure data.
