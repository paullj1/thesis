This section outlines a few additional extensions to the Adaptive Failure
Prediction Framework.  One additional type of failure that may be used to train
a predictor can be generated by under-allocating resources for the domain
controller in the sandbox hypervisor.  Under some circumstances, this may not
be considered a realistic form of failure.  However, one reason an organization
may want to implement the AFP may be that monetary resources are not available
to implement an adequately redundant domain controller and as a result, it may
be possible that an adequately sized domain controller is not an option.
Consequently, load based failure may be a realistic challenge faced by some
organizations and knowing that such a failure may occur might be valuable.

Another type of failure that the extended architecture evaluates goes one layer
deeper in analyzing the function performed by the target application.  In
addition to targeting the main library that performs authentication on a domain
controller with fault injection, the extended framework will also target the
library responsible for interacting with the user database.  In this way, the
extended AFP framework is capable of simulating a corrupted database that may
occur as a result of a corrupted sector on the disk where the database is
stored.

By adding these two additional types of failures to the data used to train a
prediction algorithm, the resulting algorithm will be able to predict a wider
range of realistic failures.
