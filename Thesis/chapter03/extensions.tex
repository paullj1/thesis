This section outlines a few additional extensions to the Adaptive Failure
Prediction Framework.  In evaluating the potential ways a domain controller
could fail, one must consider what happens when a domain controller becomes
overloaded.  As a result, one additional failure used to train a predictor is
generated by under-allocating resources for the domain controller on our
hypervisor.  Under some circumstances, this may not be considered a realistic
form of failure.  However, one reason for implementing the AFP may be that
monetary resources are not available to implement an adequately redundant
domain controller and as a result, it may be possible that an adequately sized
domain controller is not an option.  Consequently, load based failure may be a
realistic challenge faced in some organizations and knowing that the such a
failure may occur might be very valuable.

In addition to targeting the main library that performs authentication on a
domain controller, the extended framework will also target the library
responsible for interacting with the user database.  In this way, the extended
AFP framework is capable of simulating a corrupted database that may occur as a
result of a corrupted sector on the disk where the database is stored.
