\chapter{Experimental Results and Analysis} \label{chapter4}
This chapter reports results after conducting the experiments laid out in
Chapter~\ref{chapter3}.  First, common measures of performance are reviewed
followed by a short discussion of the findings.  The chapter concludes with a
detailed analysis of the outcomes of this research.

\section{Performance Measures} \label{metrics}
This section reviews the performance measures used in this chapter to
demonstrate the efficacy and quality of the models trained in this research.
These measures are commonly used in the field of machine learning to compare
and assess predictors and are taken from a survey of online failure prediction
methods written by Salfner et al.~\cite{salfnerSurvey}.

This research utilizes a technique called cross-validation in which a set of
labelled training data is broken into three parts as follows:
\begin{enumerate}
\item{Training Set:  A data set that allows a prediction model to establish and
optimize its parameters}
\item{Validation Set:  The parameters selected in the training phase are then
validated against a separate data set}
\item{Test Set:  The predictor is finally run against a final previously
unevaluated data set to assess generalizability}
\end{enumerate}
During the test phase, true positives (negatives) versus false positives
(negatives) are determined in order to compute the performance measures in
this section.  The following terms and associated abbreviations are used:
\emph{True Positive} (TP) is when failure has been predicted and then actually
occurs; \emph{False Positive} (FP) is when failure has been predicted and then
does not occur; \emph{True Negative} (TN) is when a state has been accurately
classified as non-failure prone; \emph{False Negative} is when a state has been
classified as non-failure prone and a failure occurs.

\subsection{Precision and Recall:}
Precision and recall are the most popular performance measures used when
for comparing OFP approaches.  The two are related and often times
improving precision results in reduced recall.  Precision is the number of
correctly identified failures over the number of all predicted failures.  In
other words, it reports, out of the predictions of a failure-prone state that
were made, how many were correct.  In general, the higher the precision the
better the predictor.  Precision is expressed as:

\[ Precision 
	= \dfrac{TP}{TP + FP} \in [0,1]
\]

Recall is the ratio of correctly predicted failures to the number of true
failures.  In other words, it reports, out of the actual failures that
occurred, how many the predictor classified as failure-prone.  In conjunction
with a higher precision, higher recall is indicative of a better predictor.
Recall is expressed as:

\[ Recall 
	= \dfrac{TP}{TP + FN} \in [0,1]
\]

F-Measure, as defined by~\cite{rijsbergen1979v}, is the harmonic mean of
precision and recall and represents a trade-off between the two.  A higher
F-Measure reflects a higher quality predictor.  F-Measure is expressed as:

\[ F\mhyphen Measure 
	= \dfrac{2 \cdot Precision \cdot Recall}{Precision + Recall} \in [0,1]
\]

\subsection{False Positive Rate and Specificity:}
Precision and recall do not account for true negatives (correctly predicted
non-failure-prone situations) which can bias an assessment of a predictor.  The
following performance measures take true negatives into account to help
evaluators more accurately assess and compare predictors.

False Positive Rate (FPR) is the number of incorrectly predicted failures over
the total number of predicted non-failure-prone states.  A smaller FPR reflects
a higher quality predictor.  The False Positive Rate is expressed as:

\[ \mathit{FPR}
	= \dfrac{FP}{FP + TN} \in [0,1]
\]

Specificity the number of times a predictor correctly classified a state as
non-failure-prone over all non-failure-prone predictions made.  In general,
specificity alone is not very useful since failure is rare.  Specificity is
expressed as:

\[ Specificity 
	= \dfrac{TN}{FP + TN} = 1 - False Positive Rate
\]

\subsection{Negative Predictive Value (NPV) and Accuracy:}
In some cases, we wish to show that a prediction approach can correctly
classify non-failure-prone situations.  The following performance measures 
usually can not stand alone due to the nature of failures being rare events.
In other words, a highly ``accurate'' predictor could classify a state 100\% of
the time as non-failure-prone and still fail to predict every single true
failure.

Negative Predictive Value (NPV) is the number of times a predictor correctly
classifies a state as non-failure-prone to the total number all
non-failure-prone states during which a prediction was made.  Higher quality
predictors have high NPVs.  The NPV is expressed as:

\[ \mathit{NPV}
	= \dfrac{TN}{TN + FN}
\]

Accuracy is the ratio of all correct predictions to the number of predictions
made.  Accuracy is expressed as:

\[ Accuracy 
	= \dfrac{TP + TN}{TP + FP + FN + TN}
\]

\subsection{Precision/Recall Curve:}
Much like with other predictors, many OFP approaches implement variable
thresholds to sacrifice precision for recall or vice versa.  That trade-off is
typically visualized using a precision/recall curve as shown in
Figure~\ref{fig:precisionRecallCurve}.

\figprecisionRecallCurve{3in}

Another popular visualization is the receiver operating characteristic (ROC)
curve.  By plotting true positive rate over false positive rate one is able to
see the predictors ability to accurately classify a failure.  A sample ROC
curve is shown in Figure~\ref{fig:ROC}.

\figROC{3in}

The ROC curve relationship can be further illustrated by calculating the area
under the curve (AUC).  Predictors are commonly compared using the AUC which is
calculated as follows:

$$AUC = \int_{0}^{1} \mathit{tpr}(\mathit{fpr}) \,d\,\mathit{fpr} \in [0,1],$$

\noindent
where $tpr$ = true positive rate (recall), and $fpr$ = false positive rate.  A
pure random predictor will result in an AUC of $0.5$ and a perfect predictor a
value of~$1$.  The AUC can be thought of as the probability that a predictor
will be able to accurately distinguish between a failure-prone state and a
non-failure-prone state, over the entire operating range of the predictor.

