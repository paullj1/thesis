\subsection{Data-Driven Online Failure Prediction}
The survey published by Salfner et al. covered approaches in every sub-category of the \emph{reporting} category.  Since the publication of the survey, we found approaches in two of the subcategories, \emph{pattern recognition} and \emph{classifiers}.  We therefore only cover the approaches in those sub-categories of the reporting category here.  We found some of the approaches published since Salfner's survey to be difficult to classify because they employ aspects of the other sub-categories in the \emph{reporting} category.  More specifically, many of the modern techniques seem to be a blend between the two sub-categories \emph{pattern recognition} and \emph{classifiers}.  We believe these categories have been blended because these approaches seem to follow general human intuition when looking for software failures.  In other words, we have found that cyber operators tend to look for patterns in reported errors and then classify a situation based on those patterns.  We therefore categorize these approaches as \emph{hybrid} approaches.

\subsubsection{Pattern Recognition:}
In 2006, Salfner et al. proposed an approach to predicting failures by learning patterns of similar events using a semi-Markov chain model~\cite{salfner2006}.  The model learned patterns of error reports that led to failure by mapping the reported errors to the states in the Markov chain and predicted the probability of the transition to a failure-prone state.  They tested the model using performance failures of a telecommunication system and reported a precision of 0.8, recall of 0.923, and an F-measure of 0.8571, which drastically outperformed the models to which it was compared.

Given the results, the semi-Markov Chain model is compelling however, it depends on the sequence of reported errors to remain constant in order to be effective.  Today, most software is multi-threaded or distributed so there is no guarantee that the sequence of reported errors will remain constant.  Further, the authors reported that this approach did not scale well as the complexity of the reported errors grew.

In 2007, Salfner et al. extended their previous work in~\cite{salfner2006} using semi-Markov models~\cite{salfner2007}.  They generalized the Hidden Semi-Markov process for a continuous-time model and called it the ``Generalized Hidden Semi-Markov Model (GHSMM)''  By making this generalization, the model was able to effectively predict the sequence of similar events (or in this case, errors) in the continuous time domain.  The authors then tested the model and training algorithm using telecommunication performance failure data and compared it to three other approaches.  While this GHSMM model did not perform as well as their previous work, it did outperform the models to which it was compared and more importantly did not depend on the sequence of reported errors.  In other words, this new GHSMM model predicted failure for permutations of a known failure-prone sequence making it more suited for a distributed or parallel system.

The GHSMM approach has been well received by the community, although appears to be limited in use to a single system.  Unfortunately, this approach as well as its predecessor, does not scale well and does not adapt to changes to the underlying system without retraining.

\subsubsection{Classifiers:}
In 2002, Domeniconi et al.~\cite{domeniconi2002} published a technique based on support vector machines (SVM) to classify the present state as either failure prone or not based on a window of error reports as an input vector.  As Salfner points out in~\cite{salfnerSurvey}, this SVM approach would not be useful without some sort of transformation of the input vector since the exact same sequence of error messages, rotated by one message, would not be classified as similar.  To solve this permutation challenge, the authors in~\cite{domeniconi2002} used singular value decomposition to isolate the sequence of error reports that led to a failure.

This SVM approach used training data from a production computer environment with 750 hosts over a period of 30 days.  The types of failures the system was trying to detect was the inability to route to a web-page and an arbitrary node being down.  Many approaches involving SVMs have been explored since and seem to be popular in the community~\cite{fronza2013, fulp2008, murray2005, domeniconi2002, irrera2015}.

\subsubsection{Hybrid Approaches:}
Since 2012, \emph{Fujitsu Labs} has published several papers on an approach for predicting failure in a cloud-computing environment~\cite{sonoda2012,watanabe2012,watanabe2014}.  Watanabe et al. report on findings after applying a Bayesian learning approach to detect patterns in similar log messages~\cite{watanabe2014, watanabe2012}.  Their approach abstracts the log messages by breaking them down into single words and categorizing them based on the number of identical words between multiple messages.  This hybrid approach removes the details from the messages, like node identifier, and IP address while retaining meaning of the log message.

Watanabe et al.'s hybrid approach attempts to solve the problem of underlying system changes by learning new patterns of messages in real-time.  As new messages come in, the model actively updates the probability of failure by Bayesian inference based on the number of messages of a certain type that have occurred within a certain time window.  The authors claim that their approach solves three problems: 1)  The model is not dependent upon a certain lexicon used to report errors to handle different messages from different vendors; 2)  The model does not take into account the order of messages necessarily so in a cloud environment where messages may arrive in different orders, the model is still effective; and 3)  The model actively retrains itself so manual re-training does not need to occur after system updates.  The model was then tested in a cloud environment over a ninety day period.  The authors reported a precision of 0.8 and a recall of 0.9, resulting in an F-measure of 0.847.  

In 2012, Fronza et al.~\cite{fronza2013} introduced a pattern-recognition/classifier hybrid approach that used an SVM to detect patterns in log messages that would lead to failure.  The authors used random indexing to solve the problem previously discussed of SVMs failing to classify two sequences as similar if they are offset by one error report.  The authors report that their predictor was able to almost perfectly detect non-failure conditions but was poor at identifying failures.  The authors then weighted the SVMs to account for this discrepancy by assigning a larger penalty for false negatives than false positives and had better results.
