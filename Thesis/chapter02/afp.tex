\subsection{Adaptive Failure Prediction (AFP) Framework} \label{afp}
The Adaptive Failure Prediction (AFP) Framework by Irrera, et al.
in~\cite{irrera2015} seen in Figure~\ref{fig:AFP} presents a new approach to
maintaining the efficacy of failure predictors given underlying system changes.
The authors conducted a case study implementing the framework using
virtualization and fault injection on a web server.  

\figAFP

The concept reported used past work by Irrera et
al.~\cite{irrera2013,irrera2014} to generate failure data by injecting software
faults using a tool based on G-SWFIT~\cite{gswfit} in a virtual environment for
comparing and automatically re-training predictors.  In general, the use of
simulated data is not well received by the community, however the authors
in~\cite{irrera2010,irrera2014} report evidence supporting the claim that
simulated failure data is representative of real failure data.  Further, the
authors suggest that since systems are so frequently updated and failures are
in general rare events, real failure data is often not available.  Moreover,
the literature shows that even if there is a certain type of failure in
training data and a predictor can detect and predict that type of error
accurately, it will still miss failures not present in the training data.  By
injecting the types of faults that one can expect, each failure type is
represented in the training data.

The authors then conducted a case-study using a web server and an SVM
predictor, and report their findings demonstrate their framework is able to
adapt to changes to an underlying system which would normally render a
predictor unusable.  They reported good results and concluded that the AFP is
an effective tool.  Unfortunately, the AFP is not a universal solution and
requires significant work to be implemented on a modern Microsoft Windows
enterprise network.  Furthermore, the fault load previously explored does not
completely represent all possible failures.

