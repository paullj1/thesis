\section{Extended \acrfull{AFP}} \label{chapter3}
The extended \ac{AFP} framework is an automated framework for generating
realistic failure data for the purpose of training statistical failure
prediction models.  To do this requires representative fault loads, a workload
generator, and a modern fault injection tool.  Figure~\ref{fig:highlightedAFP}
shows the original \ac{AFP} architecture with the modules updated in this work
highlighted.

This section outlines the implementation and extensions to the original
\ac{AFP} framework \citep{irrera2015}.  The \ac{AFP} framework was originally
tested on a single system running Windows XP, which has been deprecated.
Consequently, the extended \ac{AFP} framework presented has been updated to run
on the Windows Server 2008 operating system and tested against the \ac{DC}
services and an Apache web server.

\subsection{The Base \acrfull{AFP}} \label{sec:implementation}
This research builds on the experiment by \citet{irrera2015} with the following
modifications.  Since the focus of this research is on reported errors, log
messages were used to train the predictor as is done in many other recent
approaches \citep{domeniconi2002,fulp2008,salfner2007,watanabe2014}.  Instead
of only using fault injection to induce failure, this research explored three
additional fault loads: 1) third party memory leaks, 2) third party \ac{CPU}
over-utilization, and 3) heap-space corruption.  In addition to using the
\ac{SVM} model, boosted decision trees were evaluated.  Finally, in addition to
the Apache web-server, the primary target was the \ac{MS} Windows Server
running \ac{AD} Domain Services.

\subsubsection{\acrfull{AFP} Modules}
This research uses the module arrangement in the \ac{AFP}~\citep{irrera2015}
framework as an abstract system design.  The following sections detail the
virtual environment in which this architecture was constructed.

\figHighlightedAFP{0.8\textwidth}

For reference, this virtual environment was
hosted on two VMWare ESXi 5.5 hypervisors each with two 2.6 \ac{GHz} AMD
Opteron 4180 (6 cores each) \ac{CPU}s and 64 \ac{GB} memory.  The
specifications of the individual \ac{VM}s are shown in Tables~\ref{tab:hyp1},
and \ref{tab:hyp2}.

\tabHypervisorOne
\tabHypervisorTwo

\setcounter{secnumdepth}{5}

\subsubsection{Controller Hypervisor} \label{sec:controller} % 3.X.1
The controller responsibilities in this experiment were split between two
systems on a single hypervisor shown in Table~\ref{tab:hyp2}.  One system was
the \ac{MS} Windows Server responsible for workload management and fault load
management.  The other system was an Ubuntu 14.04 server that performed the
failure prediction management and event management.  \hl{In general, the
workflow can be broken down into two major phases: Execution (visualized in
Figure~\ref{fig:ExecutionPhaseHoriz}), and Training (visualized in
Figure~\ref{fig:TrainingPhase}).  In this work, retraining was initiated
manually after a software update.}  Each of these functions is
detailed in the following sections.

\figExecutionPhaseHoriz{0.8\textwidth}
\figTrainingPhase{0.5\textwidth}

\paragraph{Failure Prediction} \label{sec:failurePrediction} % 3.X.1.1
The failure prediction module predicts failure using machine learning
algorithms trained using the labelled training data generated by the rest of
this framework.  This module is either training a new predictor
because a software update occurred, or predicting failure based on log messages
and possibly other features produced by the production system.  In this
experiment, the statistical models were trained using the statistical
learning software suite \emph{R}.

\paragraph{Fault Injection Management} \label{sec:faultInjectionMgr}
This module is responsible for remotely managing the fault load used to create
realistic failure data.  It was implemented using PowerShell remote execution
on the Windows Server \ac{VM} to minimize interaction and potential biases in
the training data.  Details of the fault injection method are outlined
in~\ref{sec:faultInjectionTool}

\paragraph{Workload Management} \label{sec:workloadMgr} 
The workload management module controls the generation of computational load by
directing the sandbox workload module to create realistic work for the
virtually cloned target to accomplish.  The purpose of this module is to
accelerate the evolution of a fault into a failure.  Consider a missing
\emph{free} statement and the consequent memory leak.  A production target
server may have a large amount of available memory and the leak could be
relatively small.  To accelerate the possibility of failure occurring,
realistic work load must be generated against the sandbox clone of the
production target.

In this experiment, the management and actual load generator roles have been
divided and a new tool has been developed: \ac{D-PLG} \citep{jordan2016}.  The
\ac{D-PLG} provides a realistic and sufficient workload for implementing the
\ac{AFP} framework for a \ac{MS} \ac{DC} and web server.  The client portion of
\ac{D-PLG} was used installed on five client machines and used as the sandbox
workload generator as discussed in Section~\ref{sec:sandboxWorkload}.

\paragraph{Events Manager} \label{sec:eventsManagerMgr}
This module is responsible for receiving and managing log messages and other
events that may be used to train the failure prediction algorithm.
\citet{irrera2015} use the \ac{MS} Logman tool from the remote controller for
event management in their original case study.  Logman was configured to poll
$170$ system variables on the target machine once per second.  

Since the focus of this research is on reported errors, and the experimental
environment in this work was modelled after modern enterprise environments
where this sort of polling could produce too much data, this experiment
implemented an rsyslog server daemon and the target was configured to forward
logs to it.  Moreover, because syslog is a standard protocol, it is already in
use in many enterprise networks today.  The messages forwarded to the events
manager were then processed and added to a \ac{SQL} database for training and
prediction.  

\paragraph{Sandbox Management} \label{sec:sandboxMgr} 
The purpose of the sandbox management module is to supervise the virtual
cloning of the production system that is made when a new predictor is to be
trained.  As \citet{irrera2013,irrera2015} point out, it is typically
inappropriate to inject faults and cause failures in production systems, so a
virtual clone must be created for that purpose.  Furthermore, the
virtualization of the target process has little affect on generated data
\citep{irrera2013}.

For this experiment, the sandbox was managed manually using \ac{VM} snapshots.
After an initial stable state was configured, snapshots of every component of
the architecture were taken so that they could be reset after iterations of the
experiment.  It is important to note here that because VMWare has documented
\ac{API}s, in future work, this function could be automated.

\subsubsection{Sandbox Hypervisor} \label{sec:sandbox}
The sandbox hypervisor hosts the virtual clone of the production environment
where faults are injected and from which failure data is collected.  Cloning
the production environment ensures that the production system is not affected
and services are maintained during the training phase.  For the purposes of
this experiment, the sandbox was constructed on a single hypervisor implemented
as shown in Table~\ref{tab:hyp1}.  The following sections outline each module
within this module.

\paragraph{Fault Injection} \label{sec:faultInjectionTool} 
This module is responsible for causing the target application to fail so that
labelled failure data can be generated in a short period of time.
\citet{irrera2015} use a single tool implementing the \ac{G-SWFIT} for this
module and pointed out that this module is the most critical piece of the
\ac{AFP} implementation.  \ac{G-SWFIT} was developed by \citet{gswfit} to
emulate software failures for the purposes of software testing.  The method is
widely implemented for use in software fault injection both commercially and
academically \citep{cotroneo2012,irrera2014,natella2010,umadevi2015}. 

Unfortunately, previous \ac{G-SWFIT} tools were incapable of injecting faults
into elevated x86-64 Windows processes.  Older tools were written for Java or
x86 architectures \citep{gswfit,martins2002jaca,natella2010,sanches2011jswfit}.
For this reason, this work introduces a modernized fault injection tool capable
of injecting into x86-64 elevated system process (such as the `lsass.exe'
process).  This tool is called the
\ac{W-SWFIT}\footnote{\url{https://github.com/paullj1/w-swfit/}}, \hl{and it 
implements two of the primary faults implemented by \ac{G-SWFIT}.  These faults
simulate a missing function call as well as a missing variable assignment.}

Since many of the critical functions performed by the \ac{AD} services
processes are performed in the
`ntdsa.dll'\footnote{\url{https://technet.microsoft.com/en-us/library/cc780455(v=ws.10).aspx}}
library loaded by the `lsass.exe' process, it was the focus of fault injection.

Because of the concerns with fault
injection~\citep{cotroneo2012,kikuchi2014,natella2010}, this research generated
failure data using fault injection in conjunction with three new fault inducing
loads, covered in Section~\ref{sec:extensions}.

\paragraph{Monitoring} \label{sec:sandboxMonitoringTool} 
The purpose of this module is to capture indicators of pending failure at the
target host level so that it may be used to train a statistical prediction
model.  In this experiment, syslog was used and while it is a recognized
standard, syslog messages are not produced natively in Windows.  Fortunately,
several forwarding agents are available to translate and forward native Windows
log messages to a syslog server.  For this experiment, the \emph{Solar Winds}
syslog forwarding
tool\footnote{\url{http://www.solarwinds.com/free-tools/event-log-forwarder-for-windows/}}
was used because of its popularity in the security community and existing
presence on many enterprise networks.  The tool is a lightweight application
that simply forwards Windows events to a syslog server.

\paragraph{Sandbox Workload}  \label{sec:sandboxWorkload} 
This module creates realistic work for the target application to do before
faults are injected.  In this experiment, \ac{D-PLG} was used as the work load
generator for both the \ac{DC} and web requests.  This module was implemented
using the client portion of \ac{D-PLG} installed on five workstations and
managed by the central workload manager as discussed in
Section~\ref{sec:workloadMgr}.

\subsubsection{Target Hypervisor} \label{sec:target}
The target hypervisor was constructed as a clone of the sandbox hypervisor
shown in Table~\ref{tab:hyp1}. 

The target monitoring module was implemented exactly as the sandbox monitoring
module was, using the \emph{Solar Winds} syslog forwarding tool.  The only
modification worth noting here is that to ensure the messages were uniquely
identifiable by the controller, the hostname of the target machine was changed
after cloning.

\setcounter{secnumdepth}{3}

\subsection{Fault Load Generators} \label{sec:extensions}
This section outlines the extensions to the \ac{AFP} framework explored.
\hl{Given that fault injection can produce many faults that might otherwise
never have happened~\citep{kikuchi2014}, the next three sub-sections outline
the more targeted fault inducing loads explored in this work.}  Finally, an
outline of the changes in how data was collected from the target is presented.

\subsubsection{Under-Resourced \ac{CPU}} \label{sec:extUnderResourcedCPU}
A \ac{CPU} may become under-resourced in a few ways.  The organization
implementing the target service may not accurately anticipate the amount of
load the service may experience.  Alternatively, a third-party application
installed on the same physical machine may inadvertently consume all \ac{CPU}
time.  The result in both of these situations is the target process gets
starved of \ac{CPU} time.

This condition was simulated in two ways to accurately capture both scenarios
outlined above:  1) by downsizing the number of virtual \ac{CPU}s available
to the target \ac{VM}, and 2) by introducing a third-party application that
ran at $100\%$ \ac{CPU}.

\subsubsection{Under-Resourced Memory} \label{sec:extUnderResourcedMem}
Available memory can be limited in a few ways.  As with the under-resourced
\ac{CPU}, the implementing organization may under estimate the amount of memory
that will be needed by a server to handle the required demand.  Additionally, a
third-party application could contain a memory leak.  In both cases, the target
application may not have enough memory to accomplish the assigned work.

To test this load, this experiment created both conditions outlined above by
reducing the amount of memory available to the target \ac{VM}, and by running a
third-party application with an intentional memory leak on the target system
that slowly consumes all of the available system memory.

\subsubsection{Heap Space Corruption} \label{sec:extHeapSpaceCorrupt}
Heap-space corruption can happen in a production environment in a few ways.
First, in the Windows operating system, device drivers share critical kernel
mode libraries and have elevated permissions \citep{russinovich2009}.  If a
hardware device driver developer inadvertently writes to an area of memory not
allocated for the software, it may corrupt the memory of another process.

In this experiment, the focus of this load was on the user database.  First,
users that had been cached by the \ac{DC} process were corrupted.  Next, to
simulate a disk failure, the same user was corrupted on disk.  To do this, the
\ac{W-SWFIT} code was modified to be able to search and write anywhere in a
processes memory.  For the Apache web server, the same technique was applied to
the web pages to be served.

\subsubsection{Reported Errors} \label{sec:extReportedErrors}
This research focuses on detecting failures from reported errors instead of
system information.  These are the messages the system logs.  As pointed out by
\citet{salfnerSurvey}, a predictor only given system information is not
typically able to determine the difference between a system that is going to
fail and one that is perhaps under higher than average load.  It may be able to
pick up on \emph{undetected errors}, but there is little to distinguish those
from every day use.  Consider the \ac{DC} and a memory leak situation.
According to \citet{russinovich2009}, the \ac{MS} \ac{DC} will use as much
memory as is available to cache user credentials.  This consumption of all
available memory may appear very similar to a memory leak if system information
is all that is being recorded.
