\section{Introduction} \label{chapter1}
As dependency upon computers grows, so too do the associated risks.  Computer
systems are all around us.  Some of these systems play insignificant
roles in our lives while others are responsible for sustaining our lives.
Unfortunately, the software that controls these systems is written by humans
and consequently subject to human error.  As a result, these systems are prone
to failure with potentially catastrophic consequences.  Every day, critical
infrastructure and enterprise services depend on the reliability of computer
systems.  Being able to predict pending failure in those systems can offer
tremendous, and potentially life-saving applications in today's technologically
advanced world.  While actually being able to accurately predict failure has
unfortunately not been proven possible, there has been work over the past
several decades attempting to make educated predictions about the failure of
machines through the use of machine learning algorithms~\citep{salfnerSurvey}.
Unfortunately, much of this work has gone unused~\citep{irrera2015}.  

Failure has been defined as the result of a software fault or
error~\citep{salfnerSurvey}.  There are a number of ways to reduce the number of
errors produced by a piece of software, but the software development life-cycle
is shrinking and less time and effort are being devoted to reducing errors
before deployment~\citep{schmidt2016}.  This leaves real-time error prevention
or handling.  In recent years, it seems the recommended solution to this
problem is to make massively redundant systems that can withstand
failure~\citep{bauer2012}.  As hardware becomes more affordable, this is an
effective approach in many ways, but ultimately is still not cost efficient.
In some cases, funds may not be available to achieve this sort of redundancy.
Consequently, this research focuses on a small piece of the general field of
reliable computing: \ac{OFP}.  \ac{OFP} is the act of attempting to predict
when failures are likely so that they can be avoided.  \citet{salfnerSurvey}
outlines the recent work done in this field, much of which is not done in
production environments due to the complex and manual task of training a
prediction model.  If the underlying system changes, the efficacy of a
prediction model can be drastically reduced until it is retrained.
Furthermore, training requires access to labelled training data.  Since failure
is such a rare event, access to this type of training data may not be possible.  

\citet{irrera2015} presented the \ac{AFP} framework and case study to automate
the process of dynamically generating failure data and using it to train a
predictor after an underlying system change.  Unfortunately, the \ac{AFP}
framework required substantial changes and modernizations to be used on modern
enterprise systems.  Furthermore, the types of failures simulated within the
framework were not completely representative of real
failures~\citep{kikuchi2014}.  This research explores a modernized
implementation of the \ac{AFP} framework with a more representative fault load
on a \ac{MS} Windows Server \ac{DC}.  This implementation is then validated by
running the same experiments on the Apache web server.
