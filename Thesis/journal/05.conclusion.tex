\section{Conclusion and Future Work} \label{chapter5}
Several lines of research following this work are presented in this section.
One particularly interesting line of effort that could follow this work is to
better identify when the underlying target system has changed enough to require
retraining.  While the \ac{AFP} process is automated, it will unlikely be
necessary after every software update.  In order to avoid unnecessary use of
resources, this process could be explored.

An additional area of exploration should be to better identify how fault
injection actually affects the underlying system.  This research has shown that
in some cases, it can be extremely difficult to identify areas that will create
realistic failure conditions with any preceding indicators.  Even when
constrained, a single library can have hundreds of injection points.
Furthermore, in some cases, even when all injection points are tested, none may
lead to a realistic failure.  For this reason, the additional fault loads play
an integral role.

\subsection{Conclusion}
This research explored the use of the \ac{AFP} framework with additional fault
loads to predict failure using reported errors in the \ac{MS} \ac{DC}s.  It has
been shown that it is possible to predict failure in modern \ac{MS} enterprise
authentication architecture given a representative fault load.  Unfortunately,
at the time of writing, two out of the three fault loads introduced in this
research were not successful in generating useful failure data.  The new fault
loads are not useless however.  As was demonstrated with the \ac{SVM}
predictor, the underlying system changes can introduce or eliminate an
applications vulnerability to certain types of faults.  For this reason, if the
\ac{AFP} framework is implemented on \ac{MS} \ac{DC}s, all fault loads should
be used in the execution and training phases.

Perhaps more interestingly, fault injection, as was used in the original
\ac{AFP} framework implementation, had two outcomes: no failure occurred, or
failure occurred immediately.  In the controlled virtual environment, failure
was predictable using polled system health information, but perhaps the
indicators used to predict the failure were not actual errors but the fault
injection tool itself consuming resources.  Clearly more work must be done to
validate using fault injection alone in the \ac{AFP} framework.

In conclusion, the modified \ac{AFP} framework as presented here can be used to
effectively predict failures that might occur in a production environment and
is capable of adapting to underlying system changes using only reported errors.
To improve the efficacy of a predictor trained using this generated data, real
failure data and additional predictors can easily be integrated if available.
Finally, real failure data is difficult to obtain given how rare failure is in
modern systems.  Unfortunately, even after it is obtained, it can rapidly
become deprecated by underlying system changes.  Using the \ac{AFP} with the
fault loads introduced in this work to generate simulated failure data is the
next best thing to having real data and provides more useful predictions than
are available with no failure data.
