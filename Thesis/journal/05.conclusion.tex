\section{Conclusion and Future Work} \label{chapter5}
The three presented targeted fault-inducing loads, the new fault injection tool
\ac{W-SWFIT}, and the validation on a modern operating system extend the
\ac{AFP} framework to more effectively predict failures that might actually
occur in a production environment.  As was demonstrated with the \ac{SVM}
predictor, the underlying system changes can introduce or eliminate an
application’s vulnerability to certain types of faults.  For this reason, if
these new fault loads are implemented on \ac{MS} Windows 2008, all of them
should be used in the execution and training phases.

An additional area of exploration should be to better identify how fault
injection affects the underlying system.  This research has shown that in some
cases, it can be extremely difficult to identify areas that will create
realistic failure conditions with any preceding indicators.  Even when
constrained, a single library can have hundreds of injection points.
Furthermore, in some cases, even when all injection points are tested, none may
lead to a realistic failure.  For this reason, the additional fault loads play
an integral role.  Finally, existing prediction techniques could be applied to
this same data set to gain further insight into when a failure may occur and
possibly increase the lead time.
