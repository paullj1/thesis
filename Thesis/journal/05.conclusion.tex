\section{Conclusion and Future Work} \label{chapter5}
The presented \ac{AFP} framework extends the current \ac{AFP} framework with
additional fault loads which can be used to effectively predict failures that
might occur in a production environment, and is capable of adapting to
underlying system changes using only reported errors.  As was demonstrated with
the \ac{SVM} predictor, the underlying system changes can introduce or
eliminate an applications vulnerability to certain types of faults.  For this
reason, if the extended \ac{AFP} framework is implemented on \ac{MS} Windows
2008, all fault loads should be used in the execution and training phases.

An additional area of exploration should be to better identify how fault
injection actually affects the underlying system.  This research has shown that
in some cases, it can be extremely difficult to identify areas that will create
realistic failure conditions with any preceding indicators.  Even when
constrained, a single library can have hundreds of injection points.
Furthermore, in some cases, even when all injection points are tested, none may
lead to a realistic failure.  For this reason, the additional fault loads play
an integral role.
