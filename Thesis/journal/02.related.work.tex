\section{Overview of \ac{OFP}} \label{chapter2}
Traditionally, failure is predicted using statistical information about past
failures offline before a system is implemented.  Unfortunately, given the
complexity of modern computer systems and the infinite number of ways in which
they can be configured, this sort of offline analysis is not helpful.  \ac{OFP}
is the act of evaluating a running system in real time to make a prediction
about what the future state will be.

This section reviews current research regarding \ac{OFP} and its many
approaches to build a foundation for this research.  The rest of
this section is organized as follows.  In Section~\ref{background}, a brief
background on the topic of \ac{OFP} is given including definitions,
terminology, and measures of performance used by the community.  This
overview is then followed by a brief description of the \ac{AFP} framework.
This section then concludes with a brief summary.

\subsection{Background} \label{background}
\citet{salfnerSurvey} published a survey paper that provides a comprehensive
summary of the state of the art on the topic of \ac{OFP}.  In addition to the
review of the literature up to the point of publication, they provide a summary
of definitions and measures of performance commonly used in the community for
couching the \ac{OFP} discussion.  The remainder of this section reviews those
definitions to build a foundation for the rest of this work.

\subsubsection{\ac{PFM}} \label{pfm}
\citet{salfnerSurvey} define \ac{PFM} as the process by which faults are
handled in a proactive way, analogous with \emph{fault tolerance} and basically
consisting of four steps: \ac{OFP}, diagnosis, action scheduling, and action
execution as shown in Figure~\ref{fig:proactiveFaultManagement}.  The final
three stages of \ac{PFM} define how much lead time is required to avoid a
failure when predicted during \ac{OFP}.  \emph{Lead time} is defined as the
time between when failure is predicted and when that failure will occur.  Lead
time is one of the most critical elements of a failure prediction approach.

\figproactiveFaultManagement{0.8\textwidth}

\ac{OFP} is defined as the first step in \ac{PFM} shown in
Figure~\ref{fig:proactiveFaultManagement}.  \ac{OFP} is the act of analyzing
the running state of a system in order to predict failure in that system. Once
failure has been predicted, a fault tolerant system must determine what will
cause the failure.  This stage is called the \emph{diagnosis} stage or
``root-cause analysis'' stage.  During the \emph{diagnosis} stage, analysis
must be conducted to identify possible remediation actions.  After it is
determined what will cause a failure, a fault tolerant system must schedule a
remediation action that is either performed by an operator or done
automatically.  This stage is known as the \emph{action scheduling} stage and
normally takes as input the cost of performing an action, confidence in
prediction, effectiveness/complexity of remedy action and makes a decision
about what action to perform based on that input.  Finally, in order to avoid
failure, a system must execute the scheduled remediation action or let an
operator know which actions can be taken in a stage called \emph{action
execution}.

\paragraph{Faults, Errors, Symptoms, and Failures}
This research uses the definitions defined by \citet{avivzienis2004basic} as
interpreted and extended by \citet{salfnerSurvey} for the following terms:
failure; error (detected versus undetected); fault; and symptom.

\emph{Failure} is an event that occurs when the delivered service deviates from
correct service.  In other words, things can go wrong internally; as long as
the output of a system is what is expected, failure has not occurred.  

An \emph{error} is the part of the total state of the system that may lead to
its subsequent service failure.  \emph{Errors} are characterized as the point
when things go wrong.  Fault tolerant systems can handle errors without
necessarily evolving into failure.  There are two kinds of errors.  First, a
\emph{detected error} is an error that is reported to a logging service.  In
other words, if it can be seen in a log then it is a detected error.  Second,
\emph{undetected errors} are errors that have not been identified by an error
detector.  Undetected errors are things like memory leaks.  The error exists,
but as long as there is usable memory, it is not likely to be reported to a
logging service.  Once the system runs out of usable memory, undetected errors
will likely appear in logs and become a detected errors.  A \emph{fault} is the
hypothesized root cause of an error.  Faults can remain dormant for some time
before manifesting themselves and causing an incorrect system state.  In the
memory leak example, the missing \emph{free} statement in the source code would
be the fault.  

A \emph{symptom} is an out-of-norm behavior of a system's parameters caused by
errors, whether detected or undetected.  In the memory leak example, a possible
symptom of the error might be delayed response times due to sluggish
performance of the overall system.

\figfailureFlowDiagram{0.8\textwidth}

Figure~\ref{fig:failureFlowDiagram} illustrates how a software fault can evolve
into a failure.  Faults, errors, symptoms, and failures can be further
categorized by how they are detected also shown in
Figure~\ref{fig:failureFlowDiagram}.  \citet{salfnerSurvey} introduce a
taxonomy of \ac{OFP} approaches and classify failure prediction approaches by
the stage at which a fault is detected as it evolves into a failure: auditing,
reporting, monitoring, and tracking.  Testing is left out because it does not
help detect faults in an online sense.  

\figonlinePrediction{0.8\textwidth}

Figure~\ref{fig:onlinePrediction} demonstrates the timeline associated with
\ac{OFP}.  The parameters used by the community to define a predictor are as
follows:
\begin{itemize}
	\item{Present Time: $t$}
  \item{Lead Time: $\Delta t_{l}$, is the total time at which a predictor makes
  an assessment about the current state.}
  \item{Data Window: $\Delta t_{d}$, represents the time from which data is
  used for a predictor uses to make its assessment.}
  \item{Minimal Warning Time: $\Delta t_{w}$, is the amount of time required to
  avoid a failure if one is predicted.}
  \item{Prediction Period: $\Delta t_{p}$, is the time for which a prediction
  is valid.  As $\Delta t_{p} \rightarrow \infty$, the accuracy of the
  predictor approaches 100\% because every system will eventually fail.  As
  this happens, the usefulness of a predictor is diminished.}
\end{itemize}

As the above parameters are adjusted, predictors can become more or less
useful.  For example, it is clear that as a predictor looks further into the
future potentially increasing \emph{lead time}, confidence in its prediction is
likely to be reduced.  On the other hand, if \emph{lead time} is too small,
there will likely not be enough time to effectively take remediation action.
In general, \ac{OFP} approaches seek to find a balance between the parameters,
within an acceptable bound depending on application, to achieve the best
possible performance.

\subsection{\ac{AFP} Framework} \label{afp}
The \ac{AFP} framework by Irrera, et al.~\cite{irrera2015} shown in
Figure~\ref{fig:AFP}, presents a new approach to maintaining the efficacy of
failure predictors given underlying system changes.  The authors conducted a
case study implementing the framework using virtualization and fault injection
on a web server.  

\figAFP{0.8\textwidth}

The framework built upon past work by~\citet{irrera2013,irrera2014} to generate
failure data by injecting software faults using a tool based on
\ac{G-SWFIT}~\citep{gswfit} in a virtual environment for comparing and
automatically re-training predictors.  With the introduction of the framework,
\citet{irrera2015} report results of a case-study.  After implementing the
\ac{AFP} framework using a web server and an \ac{SVM} predictor, they report
that their findings demonstrate their framework is able to adapt to changes to
an underlying system which would normally render a predictor unusable.

In general, the use of simulated data is not well received by the community,
however \citet{irrera2010,irrera2014} report evidence supporting the claim that
simulated failure data is representative of real failure data.  Further, the
authors suggest that since systems are so frequently updated and failures are
in general rare events, real failure data is often not available.  Moreover,
the literature shows that even if there is a certain type of failure in
training data and a predictor can detect and predict that type of error
accurately, it will still miss failures not present in the training data.  By
injecting the types of faults that one can expect, each failure type is
represented in the training data.

\citet{irrera2015} reported good results and concluded that the \ac{AFP}
framework is an effective tool.  Unfortunately, the \ac{AFP} framework is not a
universal solution and requires significant work to be implemented on a modern
\ac{MS} Windows enterprise network.  Furthermore, the fault load previously
explored does not completely represent all possible
failures~\citep{kikuchi2014}.

\subsection{Summary} \label{summary}
This section covered the definitions, measures of performance, and the \ac{AFP}
framework.  There has been a tremendous amount of research surrounding the
topic of \ac{OFP} and many prediction approaches have been presented.
Unfortunately, these approaches do not appear on modern operational systems and
failures are still relatively prevalent.  \citet{irrera2015} have sought to
make predictors more adaptive to the changes in underlying systems in an effort
to make implementing existing failure predictors easier.  In this work, we
extend the \ac{AFP} framework and further generalize the approach.  
