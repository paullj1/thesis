
\documentclass{article}

\usepackage[letterpaper, total={6.5in, 8.5in}]{geometry}
\usepackage{fancyhdr}

\title{CSCE 689 Winter 2016: Project Proposal}
\author{Paul Jordan and Chip Van Patten}

\pagestyle{fancy}
\makeatletter
\let\runauthor\@author
\let\runtitle\@title
\makeatother
\lhead{\runauthor}
\rhead{\runtitle}

\begin{document}
\maketitle
\section{Proposal Abstract}
Today, network service outages can often result in mission failure.  In recent
news, network service outages have been the cause of the nationwide grounding
of commercial airlines.  In the Air Force, this kind of outage would be
catastrophic.  We depend upon the same technology as our civilian counterparts,
but our mission is National Security.  

Fortunately, an immense amount of work has gone into developing machine
learning algorithms that can warn us when failure is imminent
(\cite{salfnerSurvey,irrera2015,irrera2014,watanabe2014}) and in many cases
take action to avoid the failure.  Unfortunately, much of this work has gone
unused and further, does not work well in distributed environments.  In 2015, a
framework was proposed in~\cite{irrera2015} to make implementing failure
prediction algorithms in production environments easier.  Sadly, the framework
did not address distributed systems.  Some work has been done
in~\cite{watanabe2014,watanabe2012,sonoda2012} to address the issue of these
techniques not working in distributed systems, but these techniques require
failure to occur at least once to be effective.  For the Air Force, one failure
could have unacceptable consequences.

A significant part of the framework defined in~\cite{irrera2015} is dependent
upon being able to generate realistic load against the service under test.
Unfortunately, while much work has been done in the way of traffic generation
(\cite{avallone2004, antichi2008, molnar2013, vishwanath2009, albrecht2009,
zach2013}), none of it appears to generate the traffic we would need to load a
domain controller sufficiently.  Specifically, we need full-stack
authentication traffic and many of the generators out there only provide one
way communication.  

In our work, we intend to enable the extension of the work in~\cite{irrera2015}
by developing a tool that can generate realistic load in a distributed
environment for the purpose of training machine learning algorithms to detect
system failures in a Microsoft Windows domain.  If we are able to predict
failure with enough lead-time to enable automated failure avoidance, we can
ensure the mission critical systems stay online to enable future mission
success.

\section{Research Activities}
\subsection{Groundwork Tasks}
We'll need to first develop the tool, setup a virtual test environment, and setup
a data collection point.  The virtual environment will need to have all of the
infrastructure one might find in a domain (Domain Controller, DNS, Mail, client
machines, etc...).

\subsection{Research development tasks}
The tool we plan to write is in Powershell.  The tool must be capable of
running from a remote controller, so the controller will have to be developed as
well.

\subsection{Evaluation methods}
We will conduct an experiment with the tool to determine how realistic the
traffic generated by the tool will be.  To do this, we'll have to manually use
the client machines to conduct work tasks and collect traffic generated.  Then
we can compare the manually generated traffic with the traffic generated by our
tool.  We will also compare the log entries created by manual versus automated
use. 

\subsection{Equipment required}
We will need at least one hypervisor to setup our virtual environment (to which
we already have access), and licenses for Microsoft Windows Server and
Microsoft Windows 7 (that we have already obtained).

\subsection{Team required}
We (1st Lt Paul Jordan, and 1st Lt Chip Van Patten) plan to conduct this
research.

\section{Annotated Bibliography}
\nocite{*}

\bibliographystyle{annotate}
\bibliography{../LoadGenerator}{}

\end{document}
