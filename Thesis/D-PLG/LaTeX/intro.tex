\section{\uppercase{Introduction}}
\noindent There are many approaches to generating realistic traffic or
capturing live traffic for replay.  Unfortunately, none of these approaches are
capable of independently generating full-stack network traffic without naively
replaying previously recorded traffic, which is of limited utility in
simulating a production environment where realistic data transmission is not
necessarily confined to what's already been observed.  This research is the
result of an ongoing attempt to generalize the Adaptive Failure Prediction
(AFP) framework developed by Irrera et al. in~\cite{irrera2015}.

AFP automates the process of retraining a failure prediction algorithm after an
underlying system change by placing a target system under load before
injecting software faults to accelerate failure.  Our target system is a
Microsoft Windows active directory domain services server and, as a result, we
need to generate a full-stack authenticated session in order to sufficiently
load the service.  

In support of that research, we have developed a tool for generating several
kinds for network traffic and introduce it in this paper.  Further, we
demonstrate the validity of our tool and that our technique is generalizable
and can leverage the AFP framework in order to capture network transactions of
arbitrary arity between unbounded network components with dynamic volume,
variety, veracity, and velocity.

The rest of this paper is organized as follows.  In
Section~\ref{sec:relatedWork}, we discuss other traffic generation tools and
how ours is different.  In Section~\ref{sec:contrib}, we outline our tool and
provide specific implementation details.  In Section~\ref{sec:methodology}, we
describe the methodology used to demonstrate the efficacy of our tool.  In
Section~\ref{sec:results} we present our results after running our tool in a
virtual environment, we then conclude by outlining future work in
Section~\ref{sec:futureWork}.
