\documentclass[runningheads,a4paper]{llncs}

\usepackage{graphicx}
\graphicspath{ {../../Figures/} }

\setcounter{tocdepth}{3}
\usepackage{calc}
\usepackage{amssymb}
\usepackage{hyperref}
\usepackage{url}

\urldef{\mailsa}\path|{paul.joran,donald.vanpatten,gilbert.peterson}@afit.edu|
\urldef{\mailsb}\path|andrew.sellers@usafa.edu|
\newcommand{\keywords}[1]{\par\addvspace\baselineskip\noindent\keywordname\enspace\ignorespaces#1}

\begin{document}

\mainmatter

\title{Distributed PowerShell Load Generator (D-PLG): A New Tool for %
Dynamically Generating Network Traffic%
\thanks{The views expressed herein are solely those of the authors and do not
reflect the official policy or position of the U.S. Air Force, the Department
of Defense, or the U.S.  Government.}}

\titlerunning{D-PLG: A New Tool for Dynamically Generating Network Traffic}

\author{Paul Jordan\inst{1} \and Donald Van Patten\inst{1}\and Gilbert
Peterson\inst{1}\and Andrew Sellers\inst{2}}
%
\authorrunning{P. Jordan, D. Van Patten, G. Peterson, and A. Sellers}

\institute{
  Air Force Institute of Technology\\
  \mailsa\\
  \url{http://www.afit.edu}
  %
  \and United States Air Force Academy\\
  \mailsb\\
  \url{http://www.usafa.edu}
}

\toctitle{D-PLG: A New Tool for Dynamically Generating Network Traffic}
\tocauthor{P. Jordan, D. Van Patten, G. Peterson, and A. Sellers}

\maketitle

\begin{abstract}
%Failure in cloud infrastructure is a relatively common occurrence due to an
%array of issues.  This problem is often masked by the use of excessively
%redundant systems in virtual environments and storage area networks, but can
%still cause service interruption.  Further, properly reasoning about failure
%reduces the need for so much infrastructure redundancy and leads to more
%efficient systems.  Fortunately, many machine learning techniques have been
%presented to predict failure.  Unfortunately, much of this work has gone unused
%due to the manual and arduous maintenance these techniques require and general
%lack of labeled training data.  
%
Recently, a framework has been developed to automate the training of prediction
algorithms but has only been tested on one system.  In order to generalize the
approach a few key functions must be performed.  One of these functions is load
generation.  Unfortunately, a valid load generator has not been developed for a
Microsoft Windows active directory environment.  In this paper we introduce and
detail a tool that we have developed to make the implementation of this new
framework possible in a Microsoft domain, we present data generated by our tool
to demonstrate its efficacy, and finish with several extensions and
applications for our tool.

\keywords{Network Traffic Generator, Load Generator, Machine Learning, Online
Failure Prediction}
\end{abstract}

We as humans have always shared a curiosity about the future.  Being able to
predict events in the future offers tremendous application in todays
technologically advanced world.  While actually being able to accurately
predict the future has unfortunately not been proven possible, there has been
an enormous amount of time and energy spent over the past several decades
attempting to make educated predictions about the failure of machine in order
to avoid failures.  In this research we explore the application of a new
framework developed to automatically re-train machine learning based failure
predictors.  Failures in software based computer systems have still not been
eliminated due to the fact that software is still developed by humans and is
therefore exposed to human error.  There a number of ways to reduce the number
of errors produced by a piece of software, but the software development
life-cycle is shrinking and less time and effort are being devoted to reducing
errors before deployment.  This leaves real-time error prevention or handling.  

In recent years, it seems many of the cloud based computing companies have
attempted to solve this problem by making all of their services massively
redundant.  As hardware becomes more affordable, this is an effective approach
in many ways but ultimately is still not cost efficient.  In some cases, funds
may not be available to acheive this sort of redundancy.  Consequently, this
research focuses on a small piece of the general field of reliable computing:
online failure prediction (OFP).  OFP is the act of attempting
to predict when failures are likely so that they can be avoided.  A great deal
of work has been done in this field which we outline in chapter~\ref{chapter2},
but much of it has gone unimplemented due to the complex and manual task of
training a prediction model.  If the underlying system changes at all (which in
todays world is a common occurrence due to the aforementioned shrinking
software development life-cycle) the efficacy of a prediction model can be
drastically reduced if not rendered completely useless until it is retrained.
This research explores an implementation of a new framework for automatically
retraining a predictor after such an underlying system change.  More
specifically, we present our results after implementing this framework using a
Microsoft Windows Server domain controller.  We then apply successive software
updates until the model we have selected becomes useless and allow the
framework to re-train our predictor.

\section{Problem Statement}
Predicting and alerting on impending network service failures currently uses
thresholds and rules on discrete items in enterprise system logs.  For example,
if the central processing unit (CPU) and memory usage on a device exceeds 90\%,
then an alert may be issued.  This approach works, but only for certain types
of failures and in order to minimize the false positives, it only makes
recommendations minutes before a failure, or when the system is in an already
degraded performance mode.  To maintain network resilience, the operational
organizations responsible for communications support desperately need some
means of gaining lead-time before a service failure occurs.  

Preceding a service failure event, multiple indicators spread disparate
sources, perhaps over a long period of time, may appear in system logs.  The
log entries of interest are also quite rare compared with normal operations.
Because of these constraints, identifying failure indicators can be nearly
impossible for humans to perform.  Further, in most cases, restoring service is
more important than identifying the indicators that may or may not have
existed.  

Failure prediction can be approached in many ways. Arguably the simplest
approach is to use everyday statistical analysis to, for example, determine the
mean time between failures of specific components. The analysis of all
components making up a system can be aggregated to make predictions about that
system using a set of statistics-based or business-relevant rules.
Unfortunately, the complexity of modern architectures has outpaced such
off-line statistical-based analysis, which has driven the advancement of OFP.
OFP differs from other means of failure prediction in that it focuses on
classifying the current running state of a machine as either failure prone or
not, or in such a way that it describes the confidence in how failure prone a
system is at present~\cite{salfnerSurvey}.

Fortunately, over the past several decades many machine learning-based
approaches to identifying indications of pending failure in log messages or
\emph{reported errors} have been presented.  These data-driven approaches are
categorized in a 2010 survey paper by Salfner et al.~\cite{salfnerSurvey} on
OFP.  They categorize these data-driven approaches along with several others in
a taxonomy which we extend in this research.  They also categorize OFP under a
much broader area of study called proactive fault management (PFM).

Unfortunately, in recent years much of the work in OFP has gone unused due to
the dramatic decrease in cost and complexity involved in building
hardware-based redundant systems.  Furthermore, in most cases OFP implements
machine learning algorithms that require manual re-training after underlying
system changes.  More troubling is that these system changes are becoming more
frequent as the software development life cycle moves toward a more continuous
integration model.  To help solve these challenges, the framework presented
in~\cite{irrera2015} uses simulated faults to automatically re-train a
prediction algorithm to make implementing OFP approaches easier.  We propose to
expand the work in~\cite{irrera2015} to capture developments since its writing
and generalize it so it works for a broader class of devices.

\section{Impact of Research}
Every day, many of the Air Force's critical missions depend on our computer
infrastructure.  An essential piece of this infrastructure is the
authentication mechanisms that protect our sensitive information.
Unfortunately, the software at the core of this infrastructure is written and
maintained by humans and thus susceptible human error.  This research will
enable the Air Force and many others that use the Microsoft Enterprise
Infrastructure to accurately predict pending service outages thereby providing
lead-time in order to avoid those outages.  The result is cost savings in
personnel, equipment, but isn't limited to cost savings.  It is difficult to
quantify the risk of mission failure due to network service outage.

\section{Assumptions and Limitations}


\section{Alternate approach (from paper)}

 

\section{Related Work} \label{sec:relatedWork} 
This section is broken into two subsections.  The first briefly covers the
advances made in the field of online failure prediction and how they relate to
D-PLG, and the second details existing tools for network traffic generation.

\subsection{Online Failure Prediction}
In 2010, Salfner et al. published a survey of online failure prediction
techniques that categorized the many approaches that have been explored into a
taxonomy~\cite{salfnerSurvey}.  Many of these failure prediction techniques are
based in machine learning and require steady system states but sadly, as the
software development life-cycle has grown shorter over time, a steady system
state is no longer a guarantee.

Since the publication of Salfner et al.'s survey, it has been pointed out that
while many effective techniques for predicting failure exist, these techniques
are too difficult to maintain and consequently are not being used.  In 2015,
Irrera et al. published a framework called the Adaptive Failure Prediction
(AFP) framework for dealing with this problem that automated the process of
retraining a failure prediction algorithm after an underlying system
change~\cite{irrera2015}.  After a system change, a virtual clone of the
production system is made and then load against this clone is generated.  After
the cloned system is sufficiently loaded, faults are injected which quickly
lead to failure.  This failure is captured, labeled, and used to train a new
predictor.  The new predictor is compared against the old one and replaces it
if the new outperforms the old.

The target system for this research is a Microsoft Windows active directory
domain services server and, as a result, full-stack authenticated session
traffic is required in order to sufficiently load the service.  In this work,
we seek to enable the generalization of the AFP framework and were unable to
find a sufficient load generation tool to carry out the automated retraining of
a predictor defined in AFP.

\subsection{Network Load \& Traffic Generation}

Many tools exist for the purpose of generating network traffic.
Generally, these tools are classified into three categories: application-level,
flow-level, and packet-level generators~\cite{botta2012,zach2013}.
Application-level generators emulate traffic produced by applications on a
network, flow-level generators replicate actual traffic using statistical
modeling, and packet-level generators create and inject packets into the
network.  Network traffic generators are further classified as open- or
closed-loop.  Open-loop generators use a packet arrival model for packet
timing, whereas closed-loop generators wait for a response to a sent request
prior to sending the next request~\cite{weigle2006}.  Unfortunately, as far as
we can tell, none of the tools available generate the necessary interaction
with a deployed Microsoft Windows active directory environment necessary to
facilitate the implementation of the AFP framework.  Active directory
implements the Kerberos authentication protocol in Windows domains and due to
its cryptographic nature cannot be tested against replayed or random traffic;
rather, a sequence of valid and invalid requests and responses are necessary to
stress test this framework.  Indeed, multi-step ``handshakes'' are necessary
for rich service delivery and this capability is not realized by the current
tools with any degree of modularity or extensibility.

A brief review of the traffic generators considered when researching this
problem follows.  The Distributed Internet Traffic Generator
(D-ITG)~\cite{botta2012} is, as its name implies, a distributed traffic
generator capable of performing application, flow, and packet-level generation
using both open- and closed-loop operations -- sessions are initiated at
specific time intervals and, within each session, new requests are not sent
prior to receiving a response to the previous request.  Sadly, D-ITG currently
only supports TCP, UDP, ICMP, DNS, Telnet and VoIP which does not suit our
needs.

NTG~\cite{zach2013} is an application-level, distributed network traffic
generator which is both open- and closed-loop.  A key feature of NTG, as it
relates to our problem, is that it interacts with existing network services.
Unfortunately, it is only limited to web, mail, and multimedia
servers/services, which is insufficient for our purposes.

Swing~\cite{vishwanath2009} is a flow-level, closed-loop traffic generator that
observes live network traffic, extracts distributions from the traffic, and
generates new traffic in a manner consistent with the observed traffic
distributions.  While this tool provides the ability to generate
statistically-realistic traffic from generators to listeners across a link, the
lack of both two-way traffic and interaction with existing services
(specifically authentication services) does not satisfy the requirements for
our problem.

A final tool worth mentioning, while not a network traffic generator, is
Microsoft's Active Directory Performance Testing Tool
(ADTest)~\cite{microsoft12}.  Official Microsoft documentation is limited,
however in~\cite{bijaoui2011,morowczynski2014,suyanto2010,suyanto2010_2} we
find that ADTest assesses the ability of Microsoft 2003/2008/2012 Active
Directory Lightweight Directory Services (AD LDS) servers to add organization
units and users, and make various changes to Active Directory to aid in
developing requirements for an AD LDS deployment.  It is important to note that
Microsoft no longer supports this tool~\cite{morowczynski2014}.  Also of
importance, ADTest is not capable of testing other services that rely on active
directory domain services for authentication (e.g. RDP, SMB, etc), nor can it
be extended to do so, and is therefore insufficient for our goals.

While all of these tools are, in general, sufficient for generating traffic in
a network, they do not generate full-stack, two-way authentication that is
needed in order to sufficiently load the active directory domain services
service.  Further, these tools work by replaying traffic transactions that have
already taken place and as a result, this type of traffic cannot be used to
force any network services to do any meaningful or realistic work.  The AFP
requires the target system be placed under realistic load before injecting
faults to capture the most realistic failure data
possible~\cite{irrera2014,irrera2015}.  Due to the nature of the authentication
protocols used in enterprise domains, it is not possible to realistically
generate this traffic without valid user credentials and a dynamic
authentication session that cannot be replayed due to its cryptographic nature.
A tool to generate this type of traffic did not previously exist.

\section{Distributed PowerShell Load Generator (D-PLG)} \label{sec:contrib} 
We present D-PLG, a new tool for the generation of realistic network traffic in
a Microsoft Windows domain for the purposes of software testing or load
generation.  D-PLG can be classified as an application-level closed loop
traffic generator and is a basic Windows PowerShell script that uses native
PowerShell cmdlets for all of its functionality ensuring that the most
realistic traffic possible is generated without overburdening the client
machines used for generating load.  D-PLG offers what has not seen in other
traffic generation products or tools by making actual service requests and
producing actual challenges and responses for Windows authentication protocols.

D-PLG is written in the Windows PowerShell environment because it provides a
tremendous amount of power and flexibility to generate traffic that would
actually be generated by a users interaction with network services since the
same software applications and libraries are used.  As a result, D-PLG does
require the use of client machines. However, this work does show that
generating this type of realistic traffic is possible by utilizing only a small
number of machines, or without producing a noticeable burden on in-use client
machines.

\begin{figure}[!ht] \centering 
    \includegraphics[width=2.8in]{ConceptDiagram}
    \caption[Concept Diagram]{How each type of traffic that is generated is
    routed.  Log events are offloaded to logging service for further analysis.}
    \label{fig:conceptDiagram} 
\end{figure}

In general, the intended architecture can be seen in
Figure~\ref{fig:conceptDiagram}.  D-PLG is most effective if used by a few
client machines during idle downtimes but is developed in such a way that a
user can still use a machine that is generating load, but may notice degraded
performance depending upon how much traffic that particular client is being
asked to generate.  D-PLG is currently designed to run from one central
location, asking a configurable list of clients to produce traffic for a fixed
period of time.  

It should be noted that D-PLG implements a feature that has not been previously
seen and thus, a comparison with the existing tools is difficult.  Relevant
existing tools simply replay previously observed traffic which may be more
representative of realistic load, but are incapable of creating any real work
for cryptographic system.  Modern cryptography relies on random and dynamic
challenge-response protocols, as a result any inbound requests that are not
capable of generating dynamic challenge responses are typically dropped
immediately.

In its present form, D-PLG is comprised of three modules capable of generating
full-stack web requests, Microsoft remote desktop protocol, Microsoft server
message block (SMB) file sharing, and all associated authentication traffic.
An intended byproduct of all of this traffic is domain name system (DNS)
requests.  An important part of any active directory domain is DNS and as a
result, no load generator would be complete without performing DNS lookups.

The rest of this section outlines each of the three modules currently
implemented as well as our plans for future modules.

\subsection{Web Browsing} 
D-PLG is capable of generating full-stack web requests and presently simulates
an actual user browsing.  This module is implemented using the
`Invoke-WebRequest' PowerShell cmdlet which upon completion returns an object
representing the full document object model (DOM).  The return of this object
allows us to programmatically simulate random browsing within a returned page.
As a result, our tool is capable of generating realistic web traffic against a
web server.  This functionality is different from the functionality implemented
in many of the existing tools that only generate one-way transmission of the
web request.  As a result, this module allows users of our tool to generate
realistic load against web servers and potentially automate realistic web
application testing.

The web browser was created with minimal effort as our approach to D-PLG
emphasizes rapid generation of modern internet-based interactions.  In future
versions, we plan to implement more dynamic web browsing to facilitate the use
of D-PLG as an automated web application testing tool.  Since the entire DOM is
returned, it is possible and relatively simple to programmatically complete web
forms, and submit REST API calls in only a few lines of PowerShell code.
 
\subsection{Remote Desktop Protocol} 
Remote Desktop Protocol (RDP) is a simple protocol that allows the sharing and
remote control of a Windows desktop environment.  This module was included to
generate more authentication traffic with our active directory domain services
server as well as place load on our remote desktop services server.
Applications for this module could include network infrastructure capacity and
server sizing planning.  The module takes advantage of a modified third party
cmdlet~\cite{brasser15} which invokes a call to the native windows remote
desktop application (mstsc.exe).  Our modification only tells the cmdlet not to
present a window as to avoid interrupting an individual who may be using the
computer at the time of load generation.

Currently, the RDP module makes a full-stack remote desktop connection with an
RDP server without producing a window which can allow us to take advantage of
clients in active states.  The script then sleeps for a few seconds and then
closes the connection.  In future versions, we would like to implement some
sort of actual interaction with the RDP server like file upload or application
use.  This functionality was based on a tool previously developed by
Microsoft~\cite{microsoft09} which is no longer maintained as evidenced
here~\cite{szeto12}.

\subsection{Server Message Block (SMB) File Sharing}
D-PLG implements an SMB file sharing module that connects to a local or remote
share, creates a file in the share, fills that file with random ASCII data,
saves the file, deletes the file, and finally deletes the share.  This sequence
of operations ensures that full-stack SMB file sharing requests are utilized
and thus, causing the domain services server to authenticate the transaction
and the file sharing services server to process the data being uploaded.  This
simple module could additionally be used to ensure a file server is live before
beginning more complex operations.

Like the previous module, the SMB module was rapidly built due to the
flexibility of our framework and implemented in only fourteen lines of code.
In future versions, we plan to implement a variable amount of upload data or
allow the user to select his or her own file.  By allowing the user to upload a
custom file, this module could be used to test application aware firewall rules
to ensure certain types of files are or are not allowed to traverse a network.

\subsection{Future Modules} 
We have already implemented many core active directory domain services as a
proof of concept, but would like to point out how easy additional services
would be to implement in our script.  For example, simple message transfer
protocol (SMTP) traffic could be implemented in a single line of PowerShell
code using the `Send-MailMessage' cmdlet.  Additionally, the `Out-Printer'
cmdlet would allow for the sending of realistic full-stack network printer
traffic.  To facilitate future development, we have published D-PLG in its
current form under the MIT license on
GitHub\footnote{\url{https://github.com/paullj1/afp-dc/tree/master/D-PLG}}.

These modules demonstrate our platforms extensibility and are representative of
sophisticated network interactions that are necessary to create a performant
load generator for the tableau of modern networking services.  

\section{Methodology} \label{sec:methodology} 
This section is split into two subsections.  In the first, we describe in
detail our virtual environment.  In the following section, we detail the design
of our experiments which utilized our virtual environment.

\subsection{Virtual Environment}
The virtual environment was hosted on two VMWare ESXi 5.5 hypervisors each with
two 2.6 GHz AMD Opteron 4180 (6 cores each) CPUs and 64 GB memory.  The
individual virtual machines are detailed in Tables~\ref{fig:hyp1},
and~\ref{fig:hyp2}.  D-PLG uses cmdlets that did not exist until PowerShell
version 3.0 so each of the Microsoft (MS) Windows computers had the MS Windows
Management Framework version 4.5 installed.  The installation of this framework
also necessitated the installation of the MS .NET Framework version 4.5.  In an
enterprise environment these software frameworks would more than likely already
be installed as they are part of the service pack updates that have since been
released by Microsoft.

\begin{table}[!ht] \centering
  \caption{Hypervisor 1.}
  \begin{tabular}{ | c | l | l | c | l |}
    \hline
    Qty. & Role   & Operating System & CPU / Mem. \\ \hline\hline
    1    & DC     & Win. Server 2008 & 2 / 2 GB   \\ \hline
    5    & Client & Win. 7           & 1 / 512 MB \\ 
    \hline
  \end{tabular}
  \label{fig:hyp1}
\end{table}

\begin{table}[!ht] \centering
  \caption{Hypervisor 2.}
  \begin{tabular}{ | c | l | l | c | l |}
    \hline
    Qty. & Role & Operating System & CPU / Mem. \\ \hline\hline
    1    & RDP  & Win. Server 2008 & 1 / 4 GB   \\ \hline
    1    & Log  & Ubuntu 14.04 LTS & 1 / 1 GB   \\ 
    \hline
  \end{tabular}
  \label{fig:hyp2}
\end{table}

After installing the requisite software, each client was added to the domain
and required a few minor modifications.  First, D-PLG creates remote
`PSSessions' on each client machine and then invokes the cmdlets that have been
assembled to generate the desired load.  In order for this to happen, the
credentials of the controller must be delegated so that they may be used to
make the connections through the PSSession.  This delegation is done very
simply through the PowerShell cmdlet `Enable-WSManCredSSP'.  The final
modification was for convenience; a copy of the scripts to be executed remotely
was placed on the desktop of the Administrator user.

The domain controller had two MS Windows Server roles enabled: active directory
domain services, and domain name service (DNS).  One domain administrator
account was used for command and control traffic, and individual user accounts
were created and used for RDP and simple authentication traffic.  The RDP
server only had one MS Windows Server role enabled: remote desktop services.

The Ubuntu server was deployed and used as a central log repository for
analyzing load on the domain controller and RDP server.  The default rsyslog
application was simply configured to accept incoming connections and then the
rsyslog Windows agent was installed on the domain controller and RDP server.

D-PLG is divided into two scripts.  The first is the `LocalLoadGen' script
and is placed on each client computer.  We note here that placing the script on
the each client computer may not be ideal in a production environment and this
step could easily be automated when the controller runs.  Further, upon
completion, the script could be removed in a single PowerShell command.  The
second script `RunLoadSim' is designed to act as a command and control element
that connects to each client and executes the `LocalLoadGen' script as an
asynchronous job.  In our experiments, the command and control script was
executed from our RDP server. 

\subsection{Experiment Design} \label{sec:experimentDesign}
Two experiments were designed to test and demonstrate the efficacy of our tool
and are detailed here.  In both of the following tests, D-PLG was run five
times, where each execution consisted of five minutes of traffic generation
within our virtual environment.  The domain controller was sized based on
Microsoft's community recommendation for up to fifteen thousand users
in~\cite{mak12}.  Our goal was to produce a sufficient enough amount of traffic
to achieve the level of load that was suggested our server be able to sustain
based on how it was sized.  To determine if that goal was achieved, ESXi's
reporting tools were used to collect the relevant data in the form of packet
captures at the virtual switchports of one client machine, the terminal server,
and the domain controller.  Further data collected came from the ESXi
performance data.  After each round of our tests, the performance data were
exported from each of the hypervisors on the terminal server, one client, and
the domain controller.  In these data, CPU utilization, memory utilization,
disk operations, and network traffic are reported on twenty second intervals.
Finally, as previously stated, the rsyslog Windows Agent was used to forward
the logs from the domain controller and RDP server to an Ubuntu server.  These
log entries were then split into pieces that corresponded with each round of
the tests.

The first question we wanted to answer was, how much traffic can a PowerShell
script really generate, and is it enough to sufficiently load an enterprise
domain controller?  The first experiment was designed to answer that question.
To maximize the amount of traffic and subsequent load generated, the client
machines were only configured to make a single request.  To do this, the
`RunLoadSim' was only tasked to perform a basic authentication request to the
domain controller.  The goal was to maximize the number of authentication
requests the server could handle based on the way it was sized.  In our case,
that number was fifteen thousand users and a goal CPU utilization of 40\%.  To
prevent overburdening the client machines, we found that the highest frequency
at which these events could be created and handled was 10 per second.
Fortunately, five clients running for five minutes making ten requests per
second equated to exactly fifteen thousand requests.  

It should be noted here that the client machines used were significantly less
powerful than average desktop computers typically found in an enterprise
environment.  Each authentication event took 20 milliseconds so the maximum
number of requests per second that was observed was fifty.  As a result, this
same experimental setup can sufficiently load a domain controller sized for
seventy-five thousand users.

In the second experiment we wanted to determine how much load could be produced
without having a significant effect on the resources available to each client
machine.  We configured the clients to utilize each of the modules that are
currently implemented in D-PLG.  In each round of the test the client machines
looped continuously making an authentication request to the domain controller,
a full RDP connection, an SMB share connection, a web request to a randomly
selected URL, and finally a web request to a URL randomly selected from the
page returned by the first request.  The loop was configured to run twice per
second, however due to the high latency of the web requests, the client was not
expected to make that many requests every second of the test.  This
configuration choice was made to ensure maximum utilization when possible.

\section{Experimental Results} \label{sec:results}
In this section the results and data collected after conducting the tests
described in Section~\ref{sec:experimentDesign} are detailed.  To answer the
quantity question, the number of packets produced per second was explored as
well as CPU utilization, memory utilization, log events, and network operations
on the domain controller.  In this first round of tests, the domain controller
reported an average of 56,291 log events over each five minute test or
approximately 187 log events per second.  In addition, an average of 6,267
packets per second were captured over each of the five tests.
Figure~\ref{fig:authDCPPS} shows the distribution on the number of packets sent
and received by the domain controller for each test and tells us that the load
was consistently high throughout each test.  On the client side, as we
predicted, the load was also relatively high as seen in
Figure~\ref{fig:authClientMetrics}.

\begin{figure}[!ht] \centering
  \includegraphics[width=3in]{authDCPPS}
  \caption[Domain Controller Packets per Second]{How many packets per second
  were sent or received by the domain controller across all five rounds of the
  first test.  In each test, approximately 1.8 million packets were captured.}
  \label{fig:authDCPPS}
\end{figure}

\begin{figure}[!ht] \centering
  \includegraphics[width=3in]{authClientPPS}
  \caption[Client Packets per Second]{How many packets per second were sent or
  received by one of the clients across all five rounds of the first test.}
  \label{fig:authClientPPS}
\end{figure}

\begin{figure}[!ht] \centering
  \includegraphics[width=2.8in]{authClientMetrics}
  \caption[Test 1:  Client Metrics]{Client CPU and memory utilization during
  the first test.}
  \label{fig:authClientMetrics}
\end{figure}

\begin{figure}[!ht] \centering
  \includegraphics[width=2.8in]{authDCMetrics}
  \caption[Test 1:  Domain Controller Metrics]{Domain controller CPU and memory
  utilization during the first test.}
  \label{fig:authDCMetrics}
\end{figure}

These results validate our hypothesis.  The load generated against the domain
controller was consistently at 40\% which is exactly in-line with the amount of
load it should be expected to endure during peak business hours per the
Microsoft community recommendations for sizing~\cite{mak12}.  Unfortunately in
this case, the client machines would likely not have been usable during the
test.  Fortunately, because only five low-end are needed machines to produce
this load over a relatively short period of time, a simple solution to this
problem would be to purchase five inexpensive desktop computers for this
purpose, or conduct testing during an idle downtime.

In the second test, we were trying to find out if the client machines could
produce a sufficient amount of realistic traffic without being over burdened so
that they could be used to generate load even as individuals use them.  To
answer this question we examined CPU and memory utilization, and packets
transmitted per second with respect to the client.  The average number of
packets generated over the five minute tests was 5,499 and the remainder of
the data can be seen in Figure~\ref{fig:allModsClientMetrics}.  We also
examined these same data with respect to the domain controller and RDP server
seen in Figures~\ref{fig:allModsDCMetrics}, and~\ref{fig:allModsRDPMetrics}
respectively.

\begin{figure}[!ht] \centering
  \includegraphics[width=2.8in]{allModsClientMetrics}
  \caption[Test 2:  Client Metrics]{Client CPU and memory utilization during
  the second test.}
  \label{fig:allModsClientMetrics}
\end{figure}

\begin{figure}[!ht] \centering
  \includegraphics[width=2.8in]{allModsDCMetrics}
  \caption[Test 2:  Domain Controller Metrics]{Domain controller CPU and memory
  utilization during the second test.}
  \label{fig:allModsDCMetrics}
\end{figure}

\begin{figure}[!ht] \centering
  \includegraphics[width=2.8in]{allModsRDPMetrics}
  \caption[Test 2: RDP Metrics]{RDP server CPU and memory utilization during
  the second test.}
  \label{fig:allModsRDPMetrics}
\end{figure}

While the number of packets was relatively high, these results do not
demonstrate a sufficient amount of load on either the domain controller or RDP
server.  We suspect that this is due to the majority of the time spent during
the test retrieving web-pages.  As a result, if a proxy server or application
aware firewall is the target of this load generation, this level of traffic is
sufficient for that purpose.  These results also show that a client computer
asked to generate traffic could still be used during a test.  In future work,
we plan to test this infrastructure with non-blocking web-requests so that more
load can be placed against the local services while web requests are being
processed externally.  Alternatively, more client machines could be used during
the test.

In general, the results observed lead us to believe that our tool can be
extended and used under any circumstance where dynamic network transactions are
required between unbounded network components.  Further, these results show
that our tool will allow us to leverage the AFP framework in our future work.

\section{Future Work} \label{sec:futureWork}
There are many opportunities for improvement in D-PLG.  If D-PLG is to be used
to simulate realistic network traffic patterns that would normally be generated
by humans, much work would have to be done to balance the kinds of requests
that get made.  For example, a typical user might log in, browse the web for a
few minutes, check his or her e-mail, then maybe send an e-mail.  Currently,
D-PLG is extremely predictable with respect to what kind of request it will
make next.  The framework can be made a lot more relevant with programmatic
generation schemes such as REGEX-based pattern generation or training of input
validity via machine learning.  For the purposes of implementing the AFP
however, the level and quality of load observed is sufficient.

More configuration options could be added like the depth of a browser
simulation or having the browser simulate filling out web forms using
configurable data.  D-PLG could also allow for finer grain control over the SMB
module allowing users to select a file or specify the size of the randomly
generated file.  Most of these configuration options would be implemented in a
straightforward way.

Finally, as previously discussed, other modules which take advantage of more of
the native Windows PowerShell cmdlets like `Send-MailMessage' and
`Output-Printer' could be implemented with relative ease.

\section{Conclusion} \label{sec:conclusion}
Based on the results of our tests, we believe that D-PLG is capable of
providing sufficient load of network services in a Microsoft Windows enterprise
domain.  Our five clients were able to generate fifteen thousand authentication
requests over a five minute time period which was very near the limit that our
domain controller should be expected to  handle based on the Microsoft
community recommendations.  Further, since our configuration was based on these
recommendations, we also believe that our experiment will scale for larger
networks.  We believe that the use of client machines to produce this load is
negligible and have proven that it can be done centrally, with only a few
machines, without placing a significant burden on those client machines, and
without installing any additional software by use of the native Windows
PowerShell cmdlets.  Finally, we believe that if necessary and client machines
are used during idle down times, significant load can be generated by D-PLG.
We believe that this level of load is sufficient for our purposes of conducting
further research in the area of online failure prediction and could be used
without significant modification for other applications.

There are many established needs for having network traffic and load
generators.  We have demonstrated one important role our tool can play, but
believe that the results presented in this paper suggest that D-PLG can fill
many other needs for dynamic traffic generation.  For example, in cybersecurity
training events, traffic generators are used to simulate real traffic to mask
malicious traffic.  Other uses include equipment sizing, stress testing, and
software testing.  We believe that D-PLG can fill these needs as well and in
general can be naturally extended and used under any circumstance where dynamic
network transactions are required between unbounded network components.


\section*{Acknowledgements}
This work was supported by the the U.S. National Security Agency, National
Information Assurance Education and Training Program (Alice Shafer and Glenn
Ellisonn, Program Managers).

\vfill
\bibliographystyle{splncs03}
\bibliography{../../LoadGenerator}

\end{document}

